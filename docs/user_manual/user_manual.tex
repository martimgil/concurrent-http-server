\documentclass[a4paper,12pt]{article}

% --- Essential Packages ---
\usepackage[utf8]{inputenc}
\usepackage[T1]{fontenc}
\usepackage[english]{babel}
\usepackage{geometry}
\usepackage{graphicx}
\usepackage{hyperref}
\usepackage{listings}
\usepackage{xcolor}
\usepackage{float}

% --- Page Configuration ---
\geometry{margin=2.5cm}

% --- Code Style ---
\definecolor{codegray}{rgb}{0.95,0.95,0.95}
\lstset{
    backgroundcolor=\color{codegray},
    basicstyle=\ttfamily\small,
    breaklines=true,
    frame=single,
    numbers=none,
    captionpos=b
}

% --- Metadata ---
\title{
    \textbf{User Manual}\\
    \large Concurrent HTTP Web Server
}
\author{
    [Student Name 1] (ID: [XXXXX]) \\
    [Student Name 2] (ID: [YYYYY]) \\
    \textit{Operating Systems - University of Aveiro}
}
\date{\today}

\begin{document}

\maketitle
\tableofcontents
\newpage

% =================================================================
% 1. INTRODUCTION
% =================================================================
\section{Introduction}
This manual provides instructions for compiling, configuring, and operating the concurrent web server developed for the Operating Systems course. The server supports HTTP/1.1 requests, multi-process and multi-thread architecture, and shared memory statistics.

% =================================================================
% 2. SYSTEM REQUIREMENTS
% =================================================================
\section{System Requirements}
To run this server, the environment must meet the following requirements:
\begin{itemize}
    \item \textbf{Operating System:} Linux/Unix (e.g., Ubuntu 20.04 or later).
    \item \textbf{Compiler:} GCC with C99 support.
    \item \textbf{Libraries:} \texttt{pthread}, \texttt{rt} (Real-time extensions).
    \item \textbf{Tools:} \texttt{make}, \texttt{curl}, \texttt{ab} (Apache Bench for testing).
\end{itemize}

% =================================================================
% 3. COMPILATION
% =================================================================
\section{Compilation}
The project uses a \texttt{Makefile} for build automation. Run the following commands from the project root:

\subsection{Standard Build}
To build the production executable \texttt{bin/webserver}:
\begin{lstlisting}[language=bash]
make all
\end{lstlisting}

\subsection{Cleaning}
To remove object files and old binaries:
\begin{lstlisting}[language=bash]
make clean
\end{lstlisting}

\subsection{Debug/Test Modes}
\begin{itemize}
    \item \textbf{Helgrind (Race Detection):} \texttt{make helgrind}
    \item \textbf{ThreadSanitizer:} \texttt{make tsan}
\end{itemize}

% =================================================================
% 4. CONFIGURATION
% =================================================================
\section{Configuration}
The server behavior is controlled by the \texttt{server.conf} file located in the root directory.

\begin{table}[H]
\centering
\begin{tabular}{|l|l|p{7cm}|}
\hline
\textbf{Parameter} & \textbf{Example} & \textbf{Description} \\ \hline
PORT & 8080 & TCP port to listen on. \\ \hline
DOCUMENT\_ROOT & www & Directory containing static files. \\ \hline
NUM\_WORKERS & 4 & Number of worker processes to fork. \\ \hline
THREADS\_PER\_WORKER & 10 & Number of threads per worker. \\ \hline
MAX\_QUEUE\_SIZE & 100 & Size of the shared request buffer. \\ \hline
LOG\_FILE & server.log & Path to the access log file. \\ \hline
CACHE\_SIZE\_MB & 10 & LRU Cache size per worker (in MB). \\ \hline
\end{tabular}
\caption{Configuration Parameters}
\end{table}

% =================================================================
% 5. EXECUTION
% =================================================================
\section{Execution}

\subsection{Starting the Server}
After compilation, start the server with:
\begin{lstlisting}[language=bash]
./bin/webserver
# Or via make:
make run
\end{lstlisting}

\subsection{Graceful Shutdown}
To stop the server correctly (releasing shared memory and closing sockets):
\begin{enumerate}
    \item Press \textbf{Ctrl+C} (sends \texttt{SIGINT}) in the terminal running the server.
    \item The Master process will catch the signal, notify workers, and clean up resources.
    \item Wait for the "Server shutdown complete" message.
\end{enumerate}

% =================================================================
% 6. USAGE EXAMPLES
% =================================================================
\section{Usage Examples}

\subsection{Basic HTTP Request}
Using \texttt{curl} to fetch the index page:
\begin{lstlisting}[language=bash]
curl -v http://127.0.0.1:8080/index.html
\end{lstlisting}

\subsection{Checking Statistics}
The server prints statistics to the standard output every 30 seconds. You can also view the logs:
\begin{lstlisting}[language=bash]
tail -f server.log
\end{lstlisting}

% =================================================================
% 7. TROUBLESHOOTING
% =================================================================
\section{Troubleshooting}

\begin{itemize}
    \item \textbf{Error: Address already in use}
    \\ \textbf{Solution:} The configured PORT is occupied. Change it in \texttt{server.conf} or kill the process using it.
    
    \item \textbf{Error: Shared Memory / Semaphores failed}
    \\ \textbf{Solution:} If the server crashed previously, old shared memory objects might persist. Delete them from \texttt{/dev/shm/} or restart the PC.
\end{itemize}

\end{document}