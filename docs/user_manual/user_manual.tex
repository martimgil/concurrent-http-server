\documentclass[a4paper,12pt]{article}
%  -- Esssa estrutura foi feita em AI para servir de base ao manual de utilizador do projeto de servidor web. 
% --- Pacotes Fundamentais ---
\usepackage[utf8]{inputenc}
\usepackage[T1]{fontenc}
\usepackage[portuguese]{babel}
\usepackage{geometry}
\usepackage{graphicx}
\usepackage{hyperref}
\usepackage{listings}
\usepackage{xcolor}
\usepackage{float}

% --- Configuração da Página ---
\geometry{
    a4paper,
    left=2.5cm,
    right=2.5cm,
    top=2.5cm,
    bottom=2.5cm
}

% --- Estilo para Código (Terminal/Config) ---
\definecolor{codegray}{rgb}{0.95,0.95,0.95}
\definecolor{codeblack}{rgb}{0,0,0}

\lstset{
    backgroundcolor=\color{codegray},
    basicstyle=\ttfamily\small\color{codeblack},
    breaklines=true,
    frame=single,
    numbers=none,
    showstringspaces=false,
    tabsize=2,
    captionpos=b
}

% --- Metadados do Documento ---
\title{
    \textbf{[NOME DO PROJETO - SERVIDOR WEB]}\\
    \large Manual de Utilizador
}
\author{
    [Nome do Aluno 1] \\ NMec: [11111]
    \and
    [Nome do Aluno 2] \\ NMec: [22222]
}
\date{\today}

\begin{document}

% --- Capa e Índice ---
\maketitle
\thispagestyle{empty} % Remove número da página na capa
\newpage

\tableofcontents
\newpage

% =================================================================
% 1. INTRODUÇÃO
% =================================================================
\section{Introdução}
% [INSTRUÇÃO: Breve parágrafo a explicar o que este software faz.]
[Inserir descrição geral do servidor: suporte a HTTP/1.1, arquitetura multi-thread, objetivos principais.]

% =================================================================
% 2. REQUISITOS DO SISTEMA
% =================================================================
\section{Requisitos do Sistema}
% [INSTRUÇÃO: O que é preciso para correr isto?]

Para a correta execução do servidor, o ambiente deve cumprir os seguintes requisitos:
\begin{itemize}
    \item \textbf{Sistema Operativo:} [Inserir SO, ex: Linux Ubuntu 20.04+]
    \item \textbf{Compilador:} [Inserir versão do GCC necessária]
    \item \textbf{Bibliotecas:} [Listar bibliotecas, ex: pthread, rt]
    \item \textbf{Ferramentas:} [Listar ferramentas, ex: Make, curl, ab]
\end{itemize}

% =================================================================
% 3. COMPILAÇÃO
% =================================================================
\section{Compilação}
% [INSTRUÇÃO: Explicar os comandos do Makefile.]

O projeto utiliza um \texttt{Makefile} para automação. Utilize os seguintes comandos na raiz do projeto:

\subsection{Compilação Normal}
% [INSTRUÇÃO: Comando para gerar o executável final]
\begin{lstlisting}[language=bash]
[Inserir comando, ex: make all]
\end{lstlisting}
Este comando gera o executável em \texttt{[caminho/do/executavel]}.

\subsection{Limpeza}
% [INSTRUÇÃO: Comando para apagar ficheiros .o e binários]
\begin{lstlisting}[language=bash]
[Inserir comando, ex: make clean]
\end{lstlisting}

\subsection{Modos de Debug (Opcional)}
% [INSTRUÇÃO: Se tiveres modos extra como helgrind ou sanitizers]
\begin{itemize}
    \item \textbf{[Nome Modo 1]:} \texttt{[make target]} - [Breve descrição]
    \item \textbf{[Nome Modo 2]:} \texttt{[make target]} - [Breve descrição]
\end{itemize}

% =================================================================
% 4. CONFIGURAÇÃO
% =================================================================
\section{Configuração}
% [INSTRUÇÃO: Explicar o ficheiro server.conf]

O servidor é configurado através do ficheiro \texttt{[nome\_ficheiro.conf]}. Abaixo descrevem-se os parâmetros aceites:

\begin{table}[H]
\centering
\begin{tabular}{|l|l|p{7cm}|}
\hline
\textbf{Parâmetro} & \textbf{Valor Exemplo} & \textbf{Descrição} \\ \hline
[NOME\_PARAM\_1]   & [Valor]                & [O que este parâmetro controla] \\ \hline
[NOME\_PARAM\_2]   & [Valor]                & [O que este parâmetro controla] \\ \hline
[NOME\_PARAM\_3]   & [Valor]                & [O que este parâmetro controla] \\ \hline
[NOME\_PARAM\_4]   & [Valor]                & [O que este parâmetro controla] \\ \hline
\end{tabular}
\caption{Parâmetros de configuração}
\end{table}

% =================================================================
% 5. EXECUÇÃO
% =================================================================
\section{Execução}

\subsection{Iniciar o Servidor}
% [INSTRUÇÃO: Como arrancar o programa]
Para iniciar o servidor, execute o seguinte comando:
\begin{lstlisting}[language=bash]
[Inserir comando de execução]
\end{lstlisting}

Output esperado ao iniciar:
\begin{lstlisting}
[Inserir exemplo do texto que aparece no terminal quando o servidor arranca]
\end{lstlisting}

\subsection{Parar o Servidor (Graceful Shutdown)}
% [INSTRUÇÃO: Explicar como parar (Ctrl+C ou SIGINT) e o que acontece]
Para encerrar o servidor corretamente:
\begin{enumerate}
    \item [Passo 1: ex: Pressionar Ctrl+C]
    \item [O que o servidor faz: liberta recursos, fecha sockets, etc.]
\end{enumerate}

% =================================================================
% 6. EXEMPLOS DE UTILIZAÇÃO
% =================================================================
\section{Exemplos de Utilização}
% [INSTRUÇÃO: Demonstração de que funciona]

\subsection{Pedido Simples (Browser/Curl)}
\begin{lstlisting}[language=bash]
[Inserir comando curl exemplo]
\end{lstlisting}

\subsection{Verificação de Erros (404)}
\begin{lstlisting}[language=bash]
[Inserir comando que gera um erro propositado]
\end{lstlisting}

\subsection{Consultar Estatísticas/Logs}
% [INSTRUÇÃO: Onde ver os logs ou stats]
As estatísticas ou logs podem ser visualizados em:
\begin{itemize}
    \item [Localização 1]
    \item [Localização 2]
\end{itemize}

% =================================================================
% 7. RESOLUÇÃO DE PROBLEMAS
% =================================================================
\section{Resolução de Problemas}
% [INSTRUÇÃO: Erros comuns que o professor possa encontrar]

\begin{itemize}
    \item \textbf{Erro:} [Nome do Erro, ex: Bind failed]
    \\ \textbf{Solução:} [Como resolver, ex: Mudar a porta no config]
    
    \item \textbf{Erro:} [Outro Erro]
    \\ \textbf{Solução:} [Outra Solução]
\end{itemize}

\end{document}