\documentclass[a4paper,12pt]{article}

% --- Essential Packages ---
\usepackage[utf8]{inputenc}
\usepackage[T1]{fontenc}
\usepackage[english]{babel}
\usepackage{geometry}
\usepackage{graphicx}
\usepackage{hyperref}
\usepackage{listings}
\usepackage{xcolor}
\usepackage{float}

% --- Configuration ---
\geometry{margin=2.5cm}
\definecolor{codegray}{rgb}{0.95,0.95,0.95}
\lstset{
    backgroundcolor=\color{codegray},
    basicstyle=\ttfamily\small,
    breaklines=true,
    frame=single,
    numbers=left,
    language=C
}

% --- Metadata ---
\title{
    \textbf{Design Document}\\
    \large Multi-Threaded Web Server with IPC
}
\author{
    [Student Name 1] (ID: [XXXXX]) \\
    [Student Name 2] (ID: [YYYYY]) \\
    \textit{Operating Systems - University of Aveiro}
}
\date{\today}

\begin{document}

\maketitle
\tableofcontents
\newpage

% ----------------------------------------------------------------------
% 1. ARCHITECTURE OVERVIEW
% ----------------------------------------------------------------------
\section{Architecture Overview}
% [Instruction: Describe the Master-Worker model]
The system follows a multi-process, multi-threaded architecture designed to handle high concurrency.

\subsection{Process Model}
The application consists of a single **Master Process** and $N$ **Worker Processes**:
\begin{itemize}
    \item \textbf{Master:} Responsible for socket binding, accepting connections, and distributing them via a shared queue.
    \item \textbf{Workers:} Responsible for consuming requests from the queue and processing them using a thread pool.
\end{itemize}

\begin{figure}[H]
    \centering
    % \includegraphics[width=0.8\textwidth]{architecture_diagram.png}
    \caption{High-Level Architecture Diagram}
    \label{fig:arch}
\end{figure}

% ----------------------------------------------------------------------
% 2. SHARED DATA STRUCTURES
% ----------------------------------------------------------------------
\section{Shared Data Structures}
Inter-Process Communication (IPC) is achieved through POSIX Shared Memory.

\subsection{Memory Layout}
The main shared structure \texttt{shared\_data\_t} contains:
\begin{lstlisting}
typedef struct {
    request_queue_t queue;   // Circular buffer for socket FDs
    server_stats_t stats;    // Global statistics counters
    // ...
} shared_data_t;
\end{lstlisting}

\subsection{Circular Buffer (Producer-Consumer)}
The request queue is implemented as a bounded circular buffer (FIFO) located in shared memory. It uses \texttt{in} and \texttt{out} indices to manage insertions and removals.

% ----------------------------------------------------------------------
% 3. SYNCHRONIZATION MECHANISMS
% ----------------------------------------------------------------------
\section{Synchronization Mechanisms}
% [Instruction: Explain HOW you prevent race conditions]

\subsection{Process Synchronization (Semaphores)}
We use named POSIX semaphores to coordinate the Master and Workers:
\begin{itemize}
    \item \texttt{SEM\_MUTEX}: Ensures mutual exclusion when accessing the shared queue indices.
    \item \texttt{SEM\_ITEMS}: Counts the number of pending requests (Signals Workers).
    \item \texttt{SEM\_SPACES}: Counts the available slots in the queue (Signals Master).
\end{itemize}

\subsection{Thread Synchronization (Mutex \& Cond Vars)}
Inside each Worker process, a Thread Pool handles requests. Synchronization is achieved via:
\begin{itemize}
    \item \texttt{pthread\_mutex\_t}: Protects the internal worker queue/state.
    \item \texttt{pthread\_cond\_t}: Allows threads to sleep until new work arrives.
\end{itemize}

% ----------------------------------------------------------------------
% 4. COMPONENT DESIGN
% ----------------------------------------------------------------------
\section{Component Design}

\subsection{LRU Cache}
Each worker maintains a local Least Recently Used (LRU) cache for static files.
\begin{itemize}
    \item \textbf{Structure:} Hash map for O(1) lookups + Doubly Linked List for eviction.
    \item \textbf{Concurrency:} Protected by \texttt{pthread\_rwlock} to allow multiple simultaneous readers.
\end{itemize}

\subsection{Thread-Safe Logging}
A single log file is shared. Writes are serialized using a specific semaphore/mutex to prevent interleaved lines.

% ----------------------------------------------------------------------
% 5. SYSTEM LIFECYCLE
% ----------------------------------------------------------------------
\section{System Lifecycle}

\subsection{Initialization}
1. Load configuration.
2. Create/Open Shared Memory and Semaphores.
3. Bind and Listen on TCP Port.
4. Fork Worker processes.

\subsection{Shutdown (Graceful Exit)}
Upon receiving \texttt{SIGINT}:
1. Master stops accepting connections.
2. Signals all workers to terminate.
3. Workers join all threads and exit.
4. Master unlinks shared memory and semaphores before exiting.

% ----------------------------------------------------------------------
% 6. DIAGRAMS
% ----------------------------------------------------------------------
\section{Flowcharts}
% [Instruction: Add required flowcharts here]

\begin{figure}[H]
    \centering
    % \includegraphics[width=0.7\textwidth]{request_flow.png}
    \caption{Request Processing Flowchart}
\end{figure}

\end{document}